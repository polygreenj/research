\documentclass[11pt]{article}

\usepackage[margin=1in]{geometry}
\usepackage[T1]{fontenc}
\usepackage[utf8]{inputenc}
\usepackage{lmodern}
\usepackage{microtype}
\usepackage{setspace}
\usepackage{parskip}
\usepackage{enumitem}
\usepackage{booktabs}
\usepackage{tabularx}
\usepackage{array}
\usepackage{hyperref}
\usepackage{fancyhdr}

\hypersetup{
  colorlinks=true,
  linkcolor=black,
  urlcolor=black,
  citecolor=black
}

\setstretch{1.08}

\pagestyle{fancy}
\fancyhf{}
\lhead{Deckers Outdoor Corp (DECK)}
\rhead{10/22/2025 \,\textbar\, Patricia Y.\ Ji}
\cfoot{\thepage}

\newcolumntype{Y}{>{\raggedright\arraybackslash}X}

\begin{document}

% =========================
% Title Block
% =========================
{\LARGE \textbf{Deckers Outdoor Corp (DECK)}}\\[2pt]
{\normalsize 10/22/2025 \;\textbar\; Patricia Y.\ Ji}\\[10pt]

\textbf{Recommendation:} Long Deckers at \$101 for a 3-month horizon. Target fair price \$125--130 (base); 15\% chance to reach \$140--150 if seasonality, FX tailwinds, and corporate action align.

\vspace{10pt}

% =========================
% Company Overview
% =========================
\section*{Company Overview}
Deckers Outdoor Corporation (DECK) designs, markets, and distributes branded footwear and related apparel/accessories across lifestyle and performance categories. It operates a portfolio of global brands led by UGG (lifestyle/sheepskin heritage) and HOKA (performance running), with smaller contributions from brands like Teva; UGG and HOKA together account for roughly 95\% of company revenue. Deckers sells through a mix of wholesale partners (major retailers and specialty accounts) and direct-to-consumer channels (e-commerce and owned stores), which drives both growth and margin outcomes through changes in pricing, promotions, and channel mix.

% =========================
% Sector Thesis
% =========================
\section*{Sector Thesis}
\begin{enumerate}[leftmargin=*, label=\arabic*.]
  \item \textbf{HOKA will sustain revenue growth and limit downside risk through international sell-through strength, but margin expansion will be capped because growth is driven by wholesale and international channels rather than higher-margin U.S.\ DTC.}
  \begin{enumerate}[leftmargin=*, label=\alph*.]
    \item International sell-through is the incremental growth engine, so as long as Europe and China remain in reorder mode, HOKA’s unit growth can stay strong even if U.S.\ DTC traffic is only stable.
    \item The near-term mix shift toward wholesale and international reduces the margin upside per incremental dollar of revenue, because those channels typically carry lower gross margin and higher partner leverage than owned DTC.
    \item Reported FX translation benefits and \textasciitilde\$1.7bn net cash create an “earnings backstop” via repurchases, so even if operating margin does not expand, per-share earnings can still improve and support the multiple.
  \end{enumerate}

  \item \textbf{UGG will deliver healthy seasonal demand this year but is unlikely to drive outsized margin expansion because tougher year-over-year comps and a higher promotional cadence at major retailers and the brand site will compress average selling price recovery and reduce flow-through from incremental holiday units.}
  \begin{enumerate}[leftmargin=*, label=\alph*.]
    \item Demand can be “good” without being a margin catalyst, because the base level of holiday volume is achievable while the market is comparing against a stronger prior-year setup.
    \item A higher promotional cadence at Nordstrom, Amazon, and the brand site increases discount depth and breadth, which mechanically slows ASP recovery even when units sell through.
    \item With more promotions and tougher comps, incremental holiday units carry lower gross-profit-per-unit, so UGG can contribute to revenue and still produce limited incremental margin expansion versus what investors typically hope for in peak season.
  \end{enumerate}
\end{enumerate}

% =========================
% Macro Thesis
% =========================
\section*{Macro Thesis}
Beyond the \$125--\$130 base-case fair value, DECK can plausibly trade to \$140--\$150 into year-end if (i) the usual H2 seasonality rerating holds and (ii) the FX backdrop stays supportive, with no fundamental shock.

\begin{itemize}[leftmargin=*]
  \item \textbf{Driver 1, H2 seasonality plus rerating mechanics:} Since 2022, DECK has tended to trade materially higher in December versus June, and that pattern can repeat if holiday demand holds and the market “reprices” the name after key H2 checkpoints. The concrete mechanisms are H2 corporate actions (buybacks and/or guidance updates) plus multiple expansion after investors see Q2 results and early holiday read-through, which can lift December multiples above summer levels if fundamentals remain intact.
  \item \textbf{Driver 2, FX tailwind into late October 2025 and beyond:} Into late October 2025, markets were positioned for additional Fed easing, and the Fed did cut at the October meeting, a setup that typically pressures the USD and supports companies with meaningful international exposure. Deckers already showed this sensitivity in its June 30, 2025 10-Q: it recorded about \$9.1m of net FX-related remeasurement gains driven by favorable European and Asian moves versus the USD. With \$1.72bn cash and cash equivalents and about \$471m held by foreign subsidiaries, a weaker USD can continue to matter at the margin through translation, remeasurement, and repatriation flexibility over the next few months.
\end{itemize}

% =========================
% Key Metrics
% =========================
\section*{Key Metrics}
\begin{table}[h!]
\centering
\renewcommand{\arraystretch}{1.15}
\begin{tabularx}{\textwidth}{@{}l Y@{}}
\toprule
\textbf{Metric} & \textbf{Value} \\
\midrule
EV/EBITDA & 12.6$\times$ (FY27) \\
Forward P/E & 18--19$\times$ (FY27) \\
3-Year Anticipated EPS CAGR & \textasciitilde 10.1\% \\
Net Margin & \textasciitilde 17.6\% in FY26E \\
ROIC (TTM) & \textasciitilde 62.6\% \\
\bottomrule
\end{tabularx}
\end{table}

% =========================
% Valuation and Model Drivers
% =========================
\section*{Valuation and Model Drivers}
DCF, SOTP, and comps cluster in the mid-\$120s because the near-term debate is not “growth vs no growth,” but how much margin and multiple the market will award while tariffs/promo uncertainty is still live: consensus implies EPS rising from \$6.409 (FY26E) to \$7.771 (FY28E) with net margin normalizing from \textasciitilde 19.4\% (FY25A) to \textasciitilde 17--18\% (FY26E--FY28E), so the stock’s next-quarter upside is mechanically explained by (i) modest multiple mean reversion (mid-teens EV/EBIT or high-teens P/E), (ii) durable cash generation (FCF trajectory stays near \textasciitilde\$1B), and (iii) net-cash support, while the key cap on a $>$\$130 print is that the market typically won’t pay a materially higher multiple until it sees clean holiday sell-through and tariff mitigation translate into realized gross margin. Separately, FX can still create noise in reported opex because Deckers’ filings show corporate-level FX remeasurement gains/losses driven by EUR/Asia moves vs USD, and a large overseas cash position means FX can briefly flatter or hurt reported lines even if underlying demand is steady.

% =========================
% Catalysts
% =========================
\section*{Catalysts}
\begin{itemize}[leftmargin=*]
  \item Oct 23, 2025: FY26 Q2 results + FY26 outlook framing (sets the next-quarter multiple)
  \item Oct 24, 2025 onward: weekly retailer price/promo + sell-through checks
  \item Nov 28, 2025: Black Friday (promo intensity + channel inventory signals)
  \item Dec 1, 2025: Cyber Monday (DTC traffic conversion + discount depth)
  \item Dec 1--31, 2025: holiday sell-through window (UGG velocity, HOKA gifting/seasonal colorways)
  \item Early Jan 2026: post-holiday inventory + returns read-through (wholesale replenishment vs clearance risk)
  \item \textasciitilde Jan 29, 2026: next earnings date estimate (FY26 Q3) on UGG holiday and HOKA momentum
\end{itemize}

% =========================
% Risks
% =========================
\section*{Risks}
\begin{itemize}[leftmargin=*]
  \item \textbf{Promo-heavy holiday (25\%):} if UGG/HOKA require broader markdowns to protect sell-through through Black Friday/Cyber Monday, gross margin and operating leverage compress versus plan, pulling FY27 EPS to \textasciitilde\$6.3--6.5 and a de-risked \textasciitilde 17$\times$ multiple implies \textasciitilde\$107--111.
  \item \textbf{Tariff slippage (10\%):} if Vietnam-tariff mitigation comes in below \textasciitilde\$75m and the 2H cost step-up is larger than modeled, FY27 EPS drifts toward \textasciitilde\$6.0 and at \textasciitilde 16$\times$ the stock re-prices to roughly \textasciitilde\$96.
  \item \textbf{Macro wobble (10\%):} if U.S.\ consumer demand softens and specialty/run retail slows, DTC/wholesale units decelerate and promotional clearing rises, pushing FY27 EPS toward \textasciitilde\$5.5 and a 14--15$\times$ trough multiple yields \textasciitilde\$80--85.
  \item \textbf{FX shock (10\%):} if a risk-off move or policy surprise drives a sharp USD rally, translation and the non-core FX remeasurement tailwind reverse (including the prior SG\&A boost), reducing value by roughly \$1--2 per share versus base.
\end{itemize}

\end{document}
